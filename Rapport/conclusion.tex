\chapter*{Conclusion}
\addcontentsline{toc}{part}{Conclusion}
\label{chapter:ccl}

Tout au long de ce rapport, nous avons vu que le protocole SSL/TLS était la cible de nombreuses attaques visant à usurper une connexion ou récupérer des données sensibles. Malgré des algorithmes de chiffrement robuste, c'est souvent dans leurs implémentations que le bât blesse.

Alors que les failles sont assez vite patchées, l'interopérabilité entre les différents acteurs oblige à maintenir
une rétrocompatibilité avec les versions antérieures. Les failles mettent donc du temps à réellement disparaître.
La faille FREAK$^{\ref{chap:freak}}$ découverte il y a quelques semaines en est un bon exemple.

Des efforts sont toutefois entrepris pour abandonner les version obsolètes à l'image du protocole SSLv2 qui n'est
pratiquement plus utilisé.

Tout cela nous montre que pour se diriger vers un protocole de plus en plus sûr, il est important de maintenir une veille active dans la recherche de faille. 
