Cette partie est un état de l'art des attaques sur SSL depuis plusieurs années.
Elles peuvent être classifier en différentes catégorie :

\begin{itemize}
\item POA (Padding Oracle Attack)
\item Compression Attack
\item Timming Attack
\item Alogorithm Weakness Attack
\item Autres
\end{itemize}

Toutes ses attaques utilisent des faiblesses d'implémentations.
Ses attaques cherchent à obtenir la connaissance de données chiffrées
sans craquer l'algorithme de chiffrement. 

La plupart nécéssitent d'être en MITM (man in the middle) pour pouvoir
fonctionner. Il est parfois nécéssaire de pouvoir injecter du code 
sur le client.

Ses contraintes peuvent parraitre compliquées à mettre en place.
Dans les faits, Il est facile d'injecter du code sur un client
grâce à/à cause d'une utilisation de plus en plus courante du langage
javascript sur internet.
