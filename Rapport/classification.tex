Dans cette partie nous allons faire un état de l'art des attaques sur SSL depuis plusieurs années.
Ces dernières reposant sur des mecanismes différents, elles peuvent être classifiées dans les catégories suivantes :

\begin{itemize}
\item Oracle Attack
\item POA (Padding Oracle Attack)
\item Compression Attack
\item Timming Attack
\item Autres
\end{itemize}

Toutes ces attaques utilisent des faiblesses d'implémentations.
Elles cherchent à obtenir la connaissance de données chiffrées
sans casser l'algorithme de chiffrement. 

La plupart nécéssitent d'être en MITM (man in the middle) pour pouvoir
fonctionner. Il est parfois nécéssaire de pouvoir injecter du code 
sur le client.

Ses contraintes peuvent paraître compliquées à mettre en place.
Dans les faits, il est facile d'injecter du code sur un client
grâce à/à cause d'une utilisation de plus en plus courante du langage
javascript sur internet.
