\chapter{BREACH}
\label{chap:breach}

\section{Présentation}
\paragraph{}
BREACH\up{\cite{article:study}} a été présenté en août 2013 par Yoel Gluck, Neal Harris et Angelo Prado. Là où CRIME vérifie la taille des données compressées par TLS, là où TIME va mesurer le temps de réponse des données compressées par HTTP, BREACH va regarder la taille des réponses compressées par HTTP.

L'algorithme de compression utilisé est DEFLATE. Il utilise lui-même 2 autres algorithmes : LZ77 et le codage de Huffman.

\section{Comment ça marche}
\paragraph{}
Les pré-requis pour satisfaire cette attaque sont les mêmes que CRIME :
\begin{itemize}
  \item Pouvoir forger les requêtes du client.
  \item Se retrouver en MITM.
\end{itemize}

\paragraph{}
L'attaquant va envoyer une requête contenant dans l'en-tête 'CSRFtoken=x'. Ce bout de code sera alors présent dans le corps de la réponse, ainsi que la vraie valeur du CSRF. Lorsque le serveur web va compresser le corps de la réponse, si la valeur x est bonne, le chiffré sera plus court que si elle ne l'est pas. L'attaquant teste donc toutes les valeurs de 'x' possibles jusqu'à tomber sur un chiffré plus court que les autres.

De proche en proche, il peut retrouver la valeur entière de la donnée secrète recherchée.\\

Pour exemple, voilà ci-dessous une requête servant à trouver la valeur du CSRFtoken :

\begin{verbatim}
GET /product/?id=1234&user=CSRFtoken=x HTTP/1.1
Host: exemple.fr
(…)
\end{verbatim}

Et voilà la réponse HTTP du serveur web avant compression:

\begin{verbatim}
<form target="https://exemple.fr:443/products/catalogue.a
spx?id=1234&user=CSRFtoken=x HTTP/1.1">
(...)
<td nowrap id="tdErrLgf">
<a href="logoff.aspx?CSRFtoken=4bd634cda846fd7cb4cb0031ba
249ca2">Log Off</a>
</td>
(…)
\end{verbatim}

L'attaquant remarquera que la taille minimale du chiffré des réponses HTTP est effective pour x = '4', et pourra considérer cette valeur comme exacte pour le $1^{er}$ octet du jeton CSRF. Il pourra alors répéter l'opération pour les autres octets constituant la valeur secrète recherchée.

\section{Contre-Mesures}
\paragraph{}
Comme dans TIME, l'attaque n'aboutit qu'à condition que la donnée secrète recherchée soit également présente dans le corps de la réponse. Pour se prémunir de cette attaque, nous pouvons :

%\paragraph{}
\begin{itemize}
  \item Désactiver la compression au niveau de HTTP. Mais c'est à l'heure actuelle infaisable pour des raisons de performance et de rapidité. 
  \item Ajouter des données inutiles .
  \item Utiliser des techniques pour limiter les requêtes répètitives.
\end{itemize}
