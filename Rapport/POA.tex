
\chapter{Padding Oracle Attack}
\label{chapter:poa}

\section{Présentation}
\label{sec:pPOA}

Cette attaque\up{\cite{article:poa}} a pour objectif de retrouver le clair d'un bloc chiffré 
avec le mode CBC\up{\ref{fig:cbc}}. Elle s'appuie sur le fait que le dernier bloc chiffré
contient du padding. De plus le dernier octet de padding corespond à la
taille du padding.\\
L'oracle  déchiffre le clair et vérifie que la taille du padding est correcte. 
Il renvoie une erreur en cas de taille erronée.\\

\section{Comment ça marche}
\label{sec:ccmPOA}

Supposons maintenant que l'attaquant possède un chiffré composé de 4 blocs :
$C_1||C_2||C_3||C_4$. Le bloc $C_4$ contient un padding de taille $x$ connu par l'attaquant.
il souhaite connaître le contenu de $C_2$ et quand il interroge l'oracle, celui-ci lui dit
si la taille est correcte ou non.\\
L'attaquant peut envoyer une version contrefaite du chiffré comme suit :
$C_1||C_2||C_3'||C_4'$.\\
\begin{itemize}
\item le bloc $C_4'=C_2$
\item le bloc $C_3'= C_3$ dont le dernier octet est modifié
\end{itemize}
Soit $P_n$ le déchiffrement de $C_n$ par l'oracle et D() et E() les algorithmes
de déchiffrement et de chiffrement. 
Le bloc qui nous intéresse est $P_4'$.
\[P_4' = D(C_2) + C_3'\]
or $C_2 = E(P_2 + C_1)$ d'où

\[\Longleftrightarrow P_4' = D(E(P_2 + C_1)) + C_3'\]
\[\Longleftrightarrow P_4' = P_2 + C_1 + C_3'\]
\begin{itemize}
\item $C_1$ et $C_3'$ sont connus
\item $P_2$ est le clair recherché
\item $P_4'$ est inconnu mais l'oracle nous dit si le dernier octet est égal à $x$.
\end{itemize}
Soit k la taille des blocs et Pn[i] le ième octet du bloc n.

\[P_4'[k] = P_2[k] + C_1[k] + C_3'[k]\] si $P_4'[k] = x$ alors
\[\Longleftrightarrow x = P_2[k] + C_1[k] + C_3'[k]\]
\[\Longleftrightarrow P_2[k] = x + C_1[k] + C_3'[k]\]

Ce qui donne une équation à une seule inconnue, donc l'attaquant a trouvé $P_2[k]$.
Pour avoir $P_4' = x$, il suffit de faire varier la valeur de $C_3'[k]$. Il est donc
possible de trouver le dernier octet de $P_2$ en un maximum de 256 essais.\\

Pour trouver les autres octets, l'attaquant prend $P_4'[k] = P_2[i]$ où $i \in [1,k]$.
Avec cette méthode il est possible de retrouver tout le contenu d'un chiffré à
l'exception du premier bloc (sauf si il est possible de modifier l'IV est connu).\\

\section{Contre-Mesures}
\label{sec:cmPOA}

Cette attaque n'est pas utilisable telle quelle sur SSL car elle demande de modifier un octet
d'un bloc qui n'est pas du padding. Cela entraîne un échec du déchiffrement du bloc, le
paquet sera donc rejeté à la vérification avec la MAC. Elle est toutefois le point de
départ de nombreuses attaques sur SSL/TLS dont Poodle\up{\ref{part:poodle}}
.
