\chapter{FREAK}
\label{chap:freak}

\section{Présentation}
\label{sec:pFreak}

L'attaque FREAK (Factoring RSA-Export Key, le A peut être pour Apple ou Android), découverte
par Karthikeyan Bhargavan et l'équipe miTLS de l'inria de Paris, le 3 mars 2015.


Cette vulnérabilité résulte d'une exigence de la NSA dans les années 90. Elle imposait des chiffrements
facilement decryptables par eux aux produits vendus à l'étranger.
Ces algorithmes furent baptisés "RSA-Export" et utilisaient des clés RSA de 512 bits.


Bien entendu, ces algorithmes ne sont plus utilisés dans les implémentations actuelles, mais pour un soucis de retro-compatibilité, ils subsistent toujours dans OpenSSL.

\section{Comment ça marche}
\label{sec:ccmFreak}

Dans SSL/TLS, lors de la négociation de clé, l'algorithme de chiffrement choisi et censé être le plus
robuste commum aux deux communiquant. Si l'attaquant est en MITM, il est capable de modifier la requête ClientHello.
Cette requête envoi au serveur sa liste d'algorithmes disponibles. Cette liste est remplacé par RSA-Export. Le serveur
va alors créer une clé de taille 512 bits et celle-ci sera accepté par le client.

L'attaquant est en MITM, il peut donc observer le réseau et récupérer les clés publiques. Il lui reste alors à 
chercher la factorisation de n. Il lui sera ensuite possible de déchiffrer tout les messages chiffrés entre le
client et le serveur

Trouver la factorisation d'une clé de 512 bits peut sembler difficile mais si l'attaquant est en mesure d'utiliser
des gros serveurs de calculs, il peut le faire rapidement. A titre d'exemple, l'équipe miTLS, estime à 12 heures le temps de calcul avec un serveur Amazon EC2. Cette clé sera valide uniquement le temps de la session donc 12 heure 
peut paraitre long mais dans les faits les clés de session ne change pas souvent.

\section{Contre-Mesures}
\label{sec:cmFreak}

Cette attaque a besoin que le serveur et le client accepte d'utiliser les clés RSA-Export.

Les serveurs qui accepte ces clés représentent environ 26 \% des serveurs SSL/TLS. Une mise à jour a été faites par
TLS pour bloquer ces clés.

Au niveau des navigateurs, tous n'était pas vulnérable. Les principaux sont Safari, chrome Android et IE. 
A l'heure actuelle, chrome Android est toujours vulnérable, IE est patché et safari l'est en parti.