\chapter{FREAK}
\label{chap:freak}

L'attaque FREAK (Factoring RSA-Export Key), 
le A peut être pour Apple ou Android, découverte
par Karthikeyan Bhargavan et l'équipe miTLS de L'inria
de Paris, le 3 mars 2015.


Cette vulnérabilité résulte d'une exigence de la NSA dans
les années 90. Elle imposait des chiffrements
facilement decryptable par eux aux produits vendus à l'étranger.
Ces algorithmes furent baptisés "RSA-Export" et utilisaient
des clés RSA de 512 bits.


Bien entendu, ces algorithmes ne sont plus utilisés dans les
implémentations actuelles mais pour un soucis de retro-compatibilité, ils subsistent toujours dans OpenSSL.


Toutefois, une attaque en MITM peut forcer l'utilisation d'une
RSA-Export, il est alors possible à l'attaquant de trouver
la clé utilisé très rapidement. Les auteurs de l'attaque parlent
d'une douzaine d'heure avec une instance Amazon EC2.
