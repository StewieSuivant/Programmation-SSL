\chapter{CRIME}
\label{chapter:crime}

\section{Présentation}
\paragraph{}
CRIME (Compression Ratio Info-leak Made Easy) est une attaque visant TLS, présentée en septembre 2012 par les mêmes chercheurs que BEAST (Juliana Rizzo et Thai Duong). Cette attaque tente de récupérer de l'information lors d'une connexion sécurisée. Typiquement, l'attaquant cherche à découvrir le cookie de session de la victime afin d'usurper par la suite son identité.

\section{Comment ça marche}
\paragraph{}
Pour que l'attaque aboutisse, 2 conditions sont nécessaires :
\begin{itemize}
  \item L'attaquant doit être placé en MITM.
  \item L'attaquant doit pouvoir injecter du code (javascript) dans le navigateur de la victime.
\end{itemize}

\paragraph{}
TLS propose la compression du message. C'est à dire que les données à transmettre sont tout d'abord compressées, puis chiffrées avant d'êtres transmisent sur internet.

L'idée, c'est que l'attaquant va envoyer une requête via le navigateur de la victime, en y intégrant des données à lui.

Pour illustrer ce qui se passe, prenons 2 chaines de caractères :
\begin{itemize}
\item C1 => ATTACKERDATA:x UNKNOWNCOOKIE:y OTHERDATA»
\item C2 => ATTACKERDATA:y UNKNOWNCOOKIE:y OTHERDATA»
\end{itemize}
Nous appercevons une redondance de la lettre 'y' dans C2, ce qui implique qu'après compression, C2 sera plus courte que C1. Le chiffrement ne modifiant pas la taille de la chaine, le chiffré de C2 sera également plus court que le chiffré de C1. Il y'a donc une fuite d'information.

\paragraph{}
Le code javascript injecté va envoyer des requêtes vers le serveur cible, en y ajoutant "Cookie: sessionid=a". Le navigateur de la victime va alors intégrer le vrai cookie dans la requête (voir ci-dessous).


\begin{verbatim}
POST / HTTP/1.1
Host: bank.com
(…)
Cookie: sessionid=d8e8fca2dc0f896fd7cb4cb0031ba249
(…)
sessionid=a
\end{verbatim}


La requête est alors compressée puis chiffrée. L'attaquant regarde la taille du chiffré.
Lorsque qu'il tente d'ajouter "sessionid=d", le chiffré sera plus court (voir ci-dessous). 


\begin{verbatim}
POST / HTTP/1.1
Host: bank.com
(…)
Cookie: sessionid=d8e8fca2dc0f896fd7cb4cb0031ba249
(…)
sessionid=d
\end{verbatim}

Il suffit de répéter l'opération pour trouver les caractères suivants. L'attaquant peut ainsi reconstituer le cookie d'authentification.

\section{Contre-Mesures}
\paragraph{}
Pour corriger cette faille d'implémentation, il suffit simplement de désactiver l'option de compression TLS sur le serveur et le client.
