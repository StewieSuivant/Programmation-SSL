\chapter{CRIME}
\label{chapter:crime}

\paragraph{}
CRIME (Compression Ratio Info-leak Made Easy) est une attaque visant TLS, découverte en .... par les memes chercheurs que BEAST (Juliana Rizzo et Thai Duong). Cette attaque vise à récupérer de l'information lors d'une connexion sécurisée par TLS. Typiquement, l'attaquant cherche à découvrir le cookie de session de la victime afin d'usurper par la suite son identité.

\paragraph{}
Pour que l'attaque aboutisse, 2 conditions sont nécessaires :
\begin{itemize}
  \item L'attaquant doit etre placé en MITM (Man In The Middle).
  \item L'attaquant doit pouvoir injecter du code (javascript) dans le navigateur de la victime.
\end{itemize}

\paragraph{}
TLS propose, dans sa version ?.?, la compression. C'est à dire que les données à transmettre sont tout d'abord compressé, puis chiffrées avant d'etres transmisent sur internet.

L'idée, c'est que l'attaquant va envoyer une requete POST via le navigateur de la victime, en y intégrant des données à lui.

Pour illustrer ce qui se passe, prenons 2 chaines de caractères :
\begin{itemize}
\item C1 => ATTACKERDATA:x UNKNOWNCOOKIE:y OTHERDATA»
\item C2 => ATTACKERDATA:y UNKNOWNCOOKIE:y OTHERDATA»
\end{itemize}
On apperçoit une redondance de la lettre 'y' dans C2, ce qui implique qu'après compression, C2 sera plus courte que C1. Le chiffrement ne modifiant pas la taille de la chaine, le chiffré de C2 sera également plus court que le chiffré de C1. Il y'a donc une fuite d'information.

\paragraph{}
Le code javascript injecter va envoyer des requettes POST vers le serveur maBanque.fr, en y ajoutant 'Cookie: sessionid=a'. Le navigateur de la victime va alors intégrer le vrai cookie dans la requete (voir ci-dessous).


\begin{verbatim}
POST /anyurl HTTP/1.1
Host: bank.com
(…)
Cookie: sessionid=d8e8fca2dc0f896fd7cb4cb0031ba249
(…)
sessionid=a
\end{verbatim}


La requête est alors compressée puis chiffrée. L'attaquant regarde la taille du chiffré.
Lorsque qu'il tente d'ajouter 'sessionid=d', le chiffré sera plus court (voir ci-dessous). 


\begin{verbatim}
POST /anyurl HTTP/1.1
Host: bank.com
(…)
Cookie: sessionid=d8e8fca2dc0f896fd7cb4cb0031ba249
(…)
sessionid=d
\end{verbatim}

Il suffit de répéter l'opération pour trouver les caractères suivants. L'attaquant peut ainsi reconstituer le cookie d'authentification.

\paragraph{}
Pour corriger cette faille, il suffit simplement de désactiver l'option de copression sur le serveur et le client.
