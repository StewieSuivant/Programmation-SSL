\chapter*{Introduction}

\paragraph{}
SSL (Secure Sockets Layer) et TLS (Transport Layer Security) définissent des protocoles 
permettant de sécuriser des échanges sur internet. SSL a été initiallement développé 
par Netscape en 1995, avec la version SSL 2.0. L'IETF (Internet Engineering Task Force)
prend le relais à la suite de la SSL 3.0, et renomme le protocole TLS.
La dernière version à ce jour est TLS 1.2 sorti en 2008.

Pour des raison de compatibilité, la majeure partie de ces versions sont 
encore disponible à l'heure actuel.

\paragraph{}
TLS fonctionne en mode client-serveur, et permet de répondre à plusieurs objectifs :
\begin{itemize}
\item L'authentification du serveur et/ou du client (authentification mutuelle).
\item La confidentialité des données échangées.
\item L'intégrité des données échangées.\\
\end{itemize}


Son objectif initial et principal est de sécuriser le protocole HTTP, 
notamment pour sécuriser les paiements en ligne lors de l'essort du e-commerce. 
Aujourd'hui, son champs d'application s'est étendu : protection d'autres 
protocoles comme SMTP ou LDAP, création de VPN, \dots

%\paragraph{}
Ce protocole étant largement utilisé par les serveurs et navigateurs web 
pour la sécurisation des échanges sur internet, 
il est la cible de nombreuse attaques visant à voler une connexion, 
usurper une identité, écouter le trafic pour récolter des informations sensibles 
(mot de passe, information bancaire, Cookie de session, \dots).

%\paragraph{}
Nous allons pour commencer faire, 
le tour des attaques découvertes à partir de 2011, reposant sur différents mécanismes. 
Pour finir, nous rentrerons dans le détails d'une attaque récente, 
à savoir POODLE, verrons son implémentation et aborderons 
les contres-mesures envisageables. 

