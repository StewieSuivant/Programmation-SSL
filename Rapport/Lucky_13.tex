\chapter{Lucky 13}
\label{chapter:luck}

Cette attaque a été mise en évidence par Nadhem AlFardan and Kenny Paterson, le 4 Février 2013.

Elle est basé sur le padding Oracle Attack. Le problème de POA est qu'il n'est pas possible de déterminer
quand le padding est correct. En effet, quand celui-ci est validé, le serveur vérifie le MAC qui a été modifié par
l'attaquant. 

Dans lucky 13, l'idée est de se servir du temps de réponse pour obtenir cette information. C'est une TIMMING-ATTACK.
Elle requiert d'être en MITM et pouvoir injecter du code sur le client.

Soit un message M reçu par le serveur, \[ M = Sequence (8) || Header (5) || Data (D) || MAC (20) || PADDING (p) \]

Quand le serveur reçoit un message M, il le déchiffre et vérifie la taille du padding. 
Ensuite il doit hasher  $M = Sequence (8) || Header (5) || Data (D)$  et le comparer au MAC 
pour vérifier l'intégrité. Afin de savoir où se termine les données, il se sert de la taille du Padding.

