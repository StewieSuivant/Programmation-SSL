\chapter{Lucky 13}
\label{chapter:luck}

En cryptanalyse, une timing attack\up{\cite{article:Timming}} est une attaque par cannaux cachés dans laquelle 
un attaquant tente de compromettre un crypto-système en analysant le temps d'exécution 
de l'algorithme cryptographique. TIME, BREACH  et Lucky13 en font partie.

\section{Présentation}
\label{sec:pL13}

Cette attaque\up{\cite{article:study}} a été mise en évidence par Nadhem AlFardan and Kenny Paterson, le 4 Février 2013.

Elle est basée sur le padding Oracle Attack\up{\ref{chapter:poa}}. Le problème de POA est qu'il n'est pas possible de déterminer
quand le padding est correct. En effet, quand celui-ci est validé, le serveur vérifie le MAC qui a été modifié par l'attaquant.

Dans lucky 13, l'idée est de se servir du temps de réponse pour obtenir cette information. C'est une TIMMING-ATTACK.
Elle requiert d'être en MITM et pouvoir injecter du code sur le client.

\section{Comment ça marche}
\label{sec:ccmL13}

Soit un message M de 77 octets reçu par le serveur, \[ M = Sequence (8) || Header (5) || Data (D = 44 -p) || MAC (20) || PADDING (p) \]

Quand le serveur reçoit un message M, il le déchiffre et vérifie la taille du padding. 
Ensuite il doit hasher  $M = Sequence (8) || Header (5) || Data (D = 44 - p)$  et le comparer au MAC 
pour vérifier l'intégrité. Afin de savoir où se termine les données, il se sert de la taille du Padding.

Il y a trois cas possible : 
\begin{itemize}
\item Le padding a une taille invalide alors p vaut 0 octet. $D=44$
\item le padding a une taille valide de 0. $D=43$ 
\item la padding a une taile valide de 1.$D=42$
\end{itemize}

Le seveur va hashé son message pour le vérifier au MAC. Les hashs utilisés par SSL
ont des blocs de 64 octets, donc si un message fait plus de 64 octets, il aura deux
opération de hashs. De plus, c'est fonction ajoute au message au moins 1 octet de 
padding et 8 octets pour la longueur.

Dans notre cas, le hash fait 13 + D + 9. Ce qui donne respectivement pour nos trois cas :
66, 65 et 64 octet. Le troisième aura une seule opération de MAC et deux pour les autres.
Il est possible avec une TIMMING Attack de voir la différence. 

L'attaque Lucky 13 se base sur ça pour son attaque. Quand elle repére une différence
notable, elle peut appliquer la formule du Padding Oracle Attack.

\section{Contre-Mesures}
\label{sec:cmL13}

Une contre-mesure simple serait de rajouter des délais dans les temps de réponses,
ou bien de les uniformiser.

De toute façon, cette attaque n'est pas réalisable car la différence est difficilement observable. 
