\chapter{BEAST}
\label{chapter:Beast}

BEAST ( Browser Exploit Against SSL/TLS) est une attaque  découverte par Rizzo et Duong en septembre 2011.
Elle se base sur une vulnérabilité sur CBC trouvée en 2004 par Gregory V. Bard. De plus, elle utilise une faille dans les
navigateurs qui permet de violer le SOP (Same Origin Policy).

Le SOP est un concept des application web qui spécifie qu'une page web A ne peut accéder aux données d'une page web
B seulement si ils ont la même origine. Si elle n'était pas appliqué un site malveillant pourrait accéder aux
données d'un autre et voler des informations à un utilisateur.\\

Le mode CBC décrit plus haut~\ref{fig:cbc}, est le mode de chiffrement par bloc utilisé par SSL/TLS. 
M. Bard a montré en 2004 que le protocole était vulnérable dans le cas d'une attaque CPA (choosen Plaintext Attack).

CBC utilise un IV (initialisation vector), celui permet d'envoyer plusieur fois le même message clair
avec des chiffrés différents. Dans SSL/TLS, l'IV corespond dernier au bloc de chiffré du message précédent.
Un attaquant qui observe le réseau est donc capable de connaitre l'IV qui sera utilisé pour le prochaine message.

Dans le cas d'une attaque CPA, il est possible d'envoyer des messages clairs chosis et d'obtenir leurs chiffrés
via un oracle de chiffrement. Dans notre contexte, cela revient à demander à la cible d'envoyer des messages, et
d'observer ce qui est envoyé sur le réseau.

Un attaquant qui connait l'IV et peut créer ses messages peut essayer de retrouver le clair d'un bloc avec une 
recherche exhaustive :

\begin{itemize}
\item Soit $C = C_1 || C_2 || C_3 || C_4$, un message intercepté par l'attaquant
\item Soit $P_i = D(C_i) + C_{i-1}$ avec $1 \leq i \leq 4$ 
\item Soit $C_i = E(P_i + C_{i-1} $
\item L'attaquant cherche $C_3$
\end{itemize}

L'attaquant génére un message $P'$ comme suit :
\[ P' = C_4 + P_3' + C_2 || \dots\]
Ce qui donne le chiffré suivant :
\[ C' = E(P_3' + C_2) || \dots \]
Si $P_3' = P_3 $ alors $C_1' = C_3$, l'attaquant peut donc par recherche exhaustive retrouver le bloc $P3$, il peut
procéder ainsi pour tout les blocs. 

Il faudrait pour des blocs de taille 16 (AES) $2^{16}$ message par bloc, c'est aussi long qu'une recherche 
exhaustive sur la clé. L'attaque tel quelle n'est pas très efficace. 

Dans un message SSL/TLS, il est possible de connaitre une partie du clair. Dans le cas de l'envoie d'un cookie de 
session, le message sera de la forme : \dots Cookie: sessid= dnu6YhPb5fd0kmQ \dots

Si l'attaquant veut voler le cookie de session, il peut utiliser l'attaque ci-dessus. la cible envoie une première
requête contenant le cookie de session tel que un des blocs soit "ookie: sessid= d". Ce bloc posséde un seul octet
inconnu. Determiner cette octet avec l'attaque prendra au plus 8 essais.

Il suffit ensuite de faire envoyer à la cible le même message mais décalé comme suit : "okie: sessid= dn", et
réitérer jusqu'à reconstituer tout le cookie de session. 



