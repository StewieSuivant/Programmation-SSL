\chapter{BEAST}
\label{chapter:Beast}

\section{Présentation}
\label{sec:pBeast}

BEAST (Browser Exploit Against SSL/TLS) est une attaque  découverte par Rizzo et Duong en septembre 2011.
Elle se base sur une vulnérabilité de CBC trouvée en 2004 par Gregory V. Bard\up{\cite{article:bard}}. De plus, elle utilise une faille dans les
navigateurs qui permet de violer le SOP (Same Origin Policy).

Le SOP est un concept des applications web qui spécifie qu'une page web A ne peut accéder aux données d'une page web
B seulement si ils ont la même origine. Si elle n'était pas appliqué, un site malveillant pourrait accéder aux
données d'un autre et voler des informations à un utilisateur.

Le mode CBC décrit plus bas\up{\ref{fig:cbc}}, est le mode de chiffrement par bloc utilisé par SSL/TLS. 
M. Bard a montré en 2004 que le protocole était vulnérable dans le cas d'une attaque CPA (choosen Plaintext Attack).

CBC utilise un IV (initialisation vector), celui-ci permet chiffrer plusieurs fois le même message clair
et d'obtenir des chiffrés différents. Dans SSL/TLS, l'IV corespond au dernier au bloc de chiffré du message précédent.
Un attaquant qui observe le réseau est donc capable de connaitre l'IV qui sera utilisé pour le prochain message.

\section{Comment ça marche}
\label{sec:ccmBeast}

Dans le cas d'une attaque CPA, il est possible d'envoyer des messages à clairs chosis et d'obtenir leurs chiffrés
via un oracle de chiffrement. Dans notre contexte, cela revient à demander à la cible d'envoyer des messages, et
d'observer ce qui est envoyé sur le réseau.

Un attaquant qui connait l'IV et qui peut créer ses messages peut essayer de retrouver le clair d'un bloc avec une 
recherche exhaustive :

\begin{itemize}
\item Soit $C = C_1 || C_2 || C_3 || C_4$, un message intercepté par l'attaquant
\item Soit $P_i = D(C_i) + C_{i-1}$ avec $1 \leq i \leq 4$ 
\item Soit $C_i = E(P_i + C_{i-1}) $
\item L'attaquant cherche $P_3$
\end{itemize}

L'attaquant génère un message $P'$ comme suit :
\[ P' = C_4 + P_3' + C_2 || \dots\]
Ce qui donne le chiffré suivant :
\[ C' = E(P_3' + C_2 + C_4 + IV) || \dots  \]
or $C_4 = IV$
\[ C' = E(P_3' + C_2) || \dots \]
Si $P_3' = P_3 $ alors $C_1' = C_3$, l'attaquant peut donc par recherche exhaustive retrouver le bloc $P3$. Il peut
procéder ainsi pour tout les blocs. 

Il faudrait pour des blocs de taille 16 (AES-128) $8^{16}$ messages pour déchiffrer un bloc. Il faut autant
d'essai que pour trouver la clé donc l'attaque n'est pas utile telle quelle.\\ 

Dans un message SSL/TLS, il est possible de connaître une partie du clair. Dans le cas de l'envoie d'un cookie de 
session, le message sera de la forme : \dots Cookie: sessid=dnu6YhPb5fd0kmQ \dots

Si l'attaquant veut voler le cookie de session, il peut utiliser l'attaque ci-dessus. la cible envoie une première
requête contenant le cookie de session tel qu'un des blocs soit "Cookie: sessid=d". Ce bloc posséde un seul octet
inconnu. Determiner cette octet avec l'attaque prendra au plus 256 essais.

Il suffit ensuite de faire envoyer à la cible le même message mais décalé comme suit : "ookie: sessid=dn", et
réitérer jusqu'à reconstituer tout le cookie de session. Pour un cookie de session de taille n, il faudra maximum
n * 256 essais.

\section{Contre-Mesures}
\label{sec:cmBeast}

   L'attaque repose sur deux pré-requis : la connaissance de l'IV et la CPA. 

Il est possible de retirer la connaissance de l'IV, il suffit de rajouter de l'aléatoire dans sa géneration. 
Si l'IV n'est plus prédictible, l'attaque ne fonctionne plus. Cela est le cas à partir de TLS 1.1 et plus.

Si L'attaquant ne peut pas outrepasser le SOP, il ne plus faire de CPA. Les navigateurs ont fixé ce problème mais
il existe toujours des moyens de l'outrepasser ( cross-origin request, ...). 



